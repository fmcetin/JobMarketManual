% Created 2023-10-17 Tue 01:09
% Intended LaTeX compiler: pdflatex
\documentclass[12pt]{article}
\usepackage[utf8]{inputenc}
\usepackage[T1]{fontenc}
\usepackage{graphicx}
\usepackage{longtable}
\usepackage{wrapfig}
\usepackage{rotating}
\usepackage[normalem]{ulem}
\usepackage{amsmath}
\usepackage{amssymb}
\usepackage{capt-of}
\usepackage{hyperref}
\hypersetup{colorlinks=true, linkcolor=black, citecolor=black}
\usepackage[top=1in, bottom=1.in, left=1in, right=1in]{geometry}
\usepackage[utf8]{inputenc}
\usepackage[T1]{fontenc}
\usepackage[backend=biber,style=authoryear,natbib=true]{biblatex}
\addbibresource{../../references.bib}
\usepackage{url}
\usepackage{graphicx, adjustbox}
\usepackage{textcomp}
\usepackage{amsmath, amsfonts}
\usepackage{pdfpages}
\usepackage[version=3]{mhchem}
\usepackage{setspace}
\usepackage{indentfirst}
\usepackage{pdflscape}
\usepackage{changepage}
\usepackage{marginnote,enumitem,subfigure,rotating,fancyvrb, caption}
\author{Furkan Mustafa Cetin \\ f.m.cetin@lse.ac.uk \\\\ London School of Economics}
\date{October 2023}
\title{Practical Guide for the Academic Accounting Job Market: My Experience\footnotetext{I am profoundly grateful to my co-chairs, Andy Leone and Sugata Roychowdhury, as well as committee members Ronald Dye, Beverly Walther, and Dimitris Papanikolaou, for guiding me through this stressful process. I deeply appreciate Ferhat Akbas, Tom Hagenberg, Jung Min Kim, Doyeon Kim, Georg Rickmann, Chris Stewart, and Valerie Zhang for sharing their experiences, conducting mock interviews, and offering invaluable advice. Additionally, my gratitude extends to the accounting department team—Goldie McCarty, Kevin Lim, and Elizabeth Forest—and the Kellogg PhD Office team—Susan Jackman, Jo Ann Yablonka, and Ligia Amarei—for their unwavering logistical support.}}
\hypersetup{
 pdfauthor={Furkan Mustafa Cetin \\ f.m.cetin@lse.ac.uk \\\\ London School of Economics},
 pdftitle={Practical Guide for the Academic Accounting Job Market: My Experience\footnotetext{I am profoundly grateful to my co-chairs, Andy Leone and Sugata Roychowdhury, as well as committee members Ronald Dye, Beverly Walther, and Dimitris Papanikolaou, for guiding me through this stressful process. I deeply appreciate Ferhat Akbas, Tom Hagenberg, Jung Min Kim, Doyeon Kim, Georg Rickmann, Chris Stewart, and Valerie Zhang for sharing their experiences, conducting mock interviews, and offering invaluable advice. Additionally, my gratitude extends to the accounting department team—Goldie McCarty, Kevin Lim, and Elizabeth Forest—and the Kellogg PhD Office team—Susan Jackman, Jo Ann Yablonka, and Ligia Amarei—for their unwavering logistical support.}},
 pdfkeywords={},
 pdfsubject={},
 pdfcreator={Emacs 29.1 (Org mode 9.6.6)}, 
 pdflang={English}}
\begin{document}

\maketitle
\maketitle
\begin{abstract}
\noindent In this paper, I reflect on my journey through the 2022-2023 Academic Accounting Rookie Job Market, an experience shaped by both traditional challenges and the evolving landscape of post-COVID academia. Focusing predominantly on the US market, I also navigated interactions with European institutions active within it, leading to my appointment at the London School of Economics that I feel deeply honored and grateful for. This guide serves as a beacon for future job market candidates, shedding light on job applications, interviews, flyouts, offers and the emotional roller coaster of the hiring process. I am grateful to mentors who illuminated my path, I aim to reciprocate by distilling the wealth of advice and firsthand experiences I amassed. Nonetheless, as a reflection of personal experiences, readers are advised to absorb this content judiciously. 
\end{abstract}


\thispagestyle{empty}
\clearpage
\doublespace
\setcounter{page}{1}
\section{Introduction}
\label{sec:org6ddb705}
Hello, dear PhD student! I composed this ``manual'' primarily based on my experiences navigating the job market during my fifth year as a PhD student at the Kellogg School of Management, Northwestern University. I have titled it a ``manual'' in the hopes that it will serve as a helpful guide for you as you prepare for the job market and to give you a better appreciation of the entire process.

My intent is to share my personal experiences, provide the tips I found invaluable, and offer some advice I wish I had known earlier. However, please understand that my focus is not to comprehensively cover the entire PhD journey. For a more extensive overview of the PhD program and job market preparation, consider \citet{rouenAccountingRookieJob2017}. My manual aims to deliver concise, "pill-like" advice in bullet points, specifically tailored to navigating the job market in the post-Covid-19 era. Furthermore, it is also crucial to note that I predominantly engaged with the US market, having applied largely to US institutions, with a few European schools in the mix. Notably, these European institutions have historically been participants in the US job market. Specifically, I did not partake in the European job market held in Madrid. Thus, while some advice, especially about timing, might not resonate with other regions, adjustments can be made with ease.

I fully recognize that this phase can be incredibly stressful, but you are unlikely to find this level of interest in your research from this many people across various institutions again anytime soon. Embrace this unparalleled opportunity!

Here is a basic timeline for the job market:
\begin{enumerate}
\item Applications
\item Interviews
\item Flyouts
\item Offers
\end{enumerate}

\section{Applications}
\label{sec:orgdde4c8d}
\begin{itemize}
\item Create a professional website. As people will likely search for you online, it's beneficial to have a website that appears prominently in search results. Many academic institutions offer website templates, but there are also external options you can explore, such as \href{https://github.com/wowchemy/starter-hugo-academic}{Hugo Academic}. Alternatively, you can custom-build your own site.
\item Set a deadline for yourself to obtain final approval from your advisors by mid-October. Achieving this is ideal and less stressful option; I admire those who can accomplish it! However, if you find yourself in a situation similar to mine, you might face last-minute edits and changes. It might be acceptable (albeit stressful) if the paper's main content is agreed upon, but requires further refinement. Remember, you must have your advisors' final approval before submitting your applications and before they write recommendation letters on your behalf. Thus, you should aim to have your paper ready for your advisor no later than the third week of October, especially since some accounting departments have started setting their deadlines around the end of October or early November. But ultimately, the timing really depends on both your and your advisor's schedule. If they are teaching, they might be swamped with work and you might miss some application deadlines. Missing a few deadlines is not the end of the world (or the job market), as many schools set their deadlines in mid-November. However, it does add extra pressure.
\item Allow your letter writers at least two weeks to draft your recommendation. Typically, it is acceptable for them to upload after the stated deadline, but it remains prudent to remind them by mid-October.
\item Most open positions are listed on SSRN, but not all. Some schools post their announcements on their own websites. Therefore, if there are specific departments you are particularly interested in, it is wise to check their websites directly. You are also likely aware of ``The Spreadsheet.'' However, be cautious: some individuals may delete listings. To address this issue, a group of colleagues and I established an alternative spreadsheet. Still, creating and maintaining your own spreadsheet can help you systematically track your applications throughout the process.
\item You can find most of the open positions from SSRN, but not all of them. Some schools announce it on their website, so you might want to check the department websites you really want to apply. Most likely, you will also find ``The Spreadsheet''. However, some people don't behave and can delete the listings there. Thus, with a few friends we created another spreadsheet as well. But it would be nice to have your own spreadsheet to keep track of your applications and the process.
\item A typical application portfolio consists of your job market paper, teaching, research, and diversity, equity, and inclusion (DEI) statements, your reference letters, and a cover letter. I recommend customizing the cover letter by addressing it directly to the recruitment chair and the committee members, specifically naming the chair. If you cannot determine the name of the recruitment chair,  address the letter to the head of the department and append ``--Recruitment Committee Chair'' for clarity. Always explicitly mention the name of the school to which you are applying.
\begin{itemize}
\item Consider having your advisors and a few others review your statements. To maximize feedback, try to share these documents by mid-October.
\item Typically, you do not need to possess your reference letters. Often, you only need to provide the names and email addresses of your referees. More advice on this is provided below.
\end{itemize}
\item Seek the support of administrative staff for handling references. Ideally, you would provide the email address(es) of one staff member, who would then upload the reference letters on behalf of the faculty.\footnote{Some systems do not accept the same email address for all the reference providers and you may need to ask for additional email addresses.}  I was fortunate to have an exceptional department manager who proactively ensured that faculty submitted their reference letters to her, keeping me informed once the letters were uploaded. While you will often receive a notification from the application portal, it is crucial to personally keep track of these submissions. Be aware that I was very fortunate with a wonderful administrative team, but I have heard from friends that not all staff may be as diligent in following up on references!
\item Deciding on the number and choice of schools to apply to can be challenging. Consult with your advisors about the tiers of schools to target. In my experience, my advisors and I reviewed the list of schools I was considering. It is a deeply personal decision, but I chose to apply only to schools where I felt I would enjoy both working and living. While the allure of applying to numerous schools is strong, especially when prioritizing job security, I found it overwhelming to spread my focus too thin. I concentrated on institutions where I believed my family and I would truly thrive. Adopting this mindset was advantageous when preparing for interviews and flyouts, as I could genuinely convey my enthusiasm for joining those faculties. It is essential to note that while I share my journey, this is a personal choice and I am not advocating any specific strategy.
\end{itemize}

\section{Interviews}
\label{sec:org5463bb3}
\begin{itemize}
\item After receiving applications, schools usually contact candidates via email, primarily after Thanksgiving and through December (even at the end of December post-Covid era). More often than not, they provide a link for you to reserve a suitable interview time. It is advisable to act swiftly to secure a convenient slot. However, there are instances where they might offer limited or no choices. It is customary for them to share the names of the faculty members who will attend the interview. If they do not, it is acceptable to politely inquire. Interviews might involve just a couple of faculty members, or there could be so many in the room that it's challenging to distinguish individual faces.
\item Especially after the first interview invitation, seek out faculty members for mock interviews. The more, the better! Especially solicit advice from junior faculty members; they recently underwent the same experience.
\begin{itemize}
\item Prepare a list of potential questions alongside your answers. While you should not aim to memorize them, the act of writing can significantly aid your preparation. If uncertain about how your answers come across, seek feedback from your advisors. As you participate in interviews and flyouts, jot down any new questions and your responses to them. As a result, this list should evolve over time.
\end{itemize}
\item Virtual interviews have become standard since the advent of COVID-19. However, it's essential to present yourself professionally. Casual or wrinkled attire might be perceived as a lack of seriousness. For the illusion of eye contact, focus on the camera. Ensure your lighting and audio quality are optimal, so interviewers can clearly see and hear you. Conduct several of your mock interviews online to receive feedback on this aspect. You might want to consider investing in high-quality equipment, such as a webcam, lighting, and microphone.
\item A typical interview commences with ``tell us about yourself and your research.'' For virtual interviews, some institutions may allow the use of slides, though I generally advise against it. Sum up your overarching research interests and relate them to your job market paper. Highlight the primary research questions, briefly explain your identification strategy, and state your main findings. This should ideally be wrapped up in about three minutes. Dwelling too long on this can reduce the valuable interaction time with the faculty.
\begin{itemize}
\item Prepare varying lengths of your research summary: 1-minute, 3-minute, and 5-minute versions. Repeatedly practice delivering each to ensure comfort and confidence. A scenario might arise where you are in an elevator with a key individual from a dream institution; the 1-minute summary can be handy then.
\item They may ask about your other papers and be prepared to briefly discuss them.
\end{itemize}
\item Maintain enthusiasm about your research throughout all interviews. Regardless of the number of times you discuss your research on that day (might be the tenth time!), sustain high energy and passion. Even in your last interview of the day, your excitement must not wane. Otherwise the faculty will think, ``if they are not passionate, why should we be?''
\item Avoid excessive time summarizing your work. The interview should resemble a good tennis match, featuring significant back-and-forth interactions with the faculty.
\begin{itemize}
\item Monitor your time in mock interviews to avoid overextending.
\item Strive for an interactive discussion, allowing faculty to interject with questions.
\end{itemize}
\item When faced with a question, prioritize clarity over a quick response. It is essential to fully grasp the question before answering. If in doubt, it is better to ask for clarification or repeat the query back to ensure you have understood correctly. Remember, seeking clarification often reflects thoroughness and attentiveness, while misunderstanding can leave a less favorable impression.
\begin{itemize}
\item Start your response with a succinct summary before diving into a more detailed explanation.
\item On occasion, it is okay to admit if you are unsure of an answer, expressing a willingness to contemplate it further. However, such admissions should be rare, ideally no more than once during an interview. Some questions are strategically posed to gauge the depth of your understanding and thought process.
\end{itemize}
\item A common inquiry is whether you have any questions. Although I don’t have a comprehensive list of questions, refrain from asking about teaching load or research budget in the first-stage interview (unless extremely necessary). Such questions might be more appropriate during flyouts. as they would be more appreciate questions for the flyouts. Try asking specific questions about the school that would signal your genuine interest. Asking about their hiring timeline is reasonable, but I would refrain from probing about the number of interviewees or flyout candidates. Generally, my guiding principle is to concentrate on factors within my control.
\item Post-interview, dispatch thank-you emails to the participating faculty. It might seem redundant, especially if you've expressed gratitude at the interview's conclusion. However, it's essential to ensure each email reflects the nuances of the specific interview. Remember, these faculty members have dedicated a substantial amount of their time to the hiring process, including the duration allocated to you. For some, such gestures matter significantly, interpreting it as an indication of your keenness in their institution.
\item Send thank-you emails to the faculty who interviewed you. While it might seem redundant, especially if you have already expressed gratitude at the end of the interviews, or you might be very considerate and do not wish to consume more of the faculty's time, it is still a thoughtful gesture. Firstly, recognize that these faculty members have dedicated a significant portion of their time to the hiring process, and more importantly, to your interview. Secondly, for some faculty members, this gesture truly matters! Avoid generic emails; instead, tailor each message based on the specifics of your interaction. They may interpret it as an indication of your interest in their institution. Regrettably, I was initially unaware that sending thank-you emails was a widespread practice in the accounting job market. I mistakenly thought it would be unnecessary and burdensome for faculty. Yet, it is important to realize that for some, these gestures carry weight.
\end{itemize}

\section{Flyouts}
\label{sec:org4bd3901}
\begin{itemize}
\item Some schools communicate only with candidates to whom they extend a flyout invitation, while others inform all interviewees about the flyout decisions. Additionally, some institutions might reach out via phone call, so it is advisable to answer any unfamiliar numbers during this period!
\item Always opt for fully refundable flight tickets. Given the likelihood of a packed schedule, you may need to alter flight arrangements as new invitations come in. Schools may also request a change in the flyout date. Without a refundable ticket, altering your flight can be a cumbersome and often painful process. While it is prudent to be cost-conscious, do not overly stress about the ticket prices. Schools will typically reimburse you as long as your ticket is in economy class. In some cases, if the flight duration exceeds a certain number of hours, schools might even cover a business class ticket!
\item In the event of numerous flyouts, it may be worthwhile to explore the perks associated with your credit card or open up a new one. Some cards offer complimentary access to airport lounges or priority boarding, benefits that can significantly enhance your travel experience.
\item Stay organized with your tickets, receipts, and other documentation. Process the reimbursement immediately after your flyout. Extend your gratitude to the organizers for coordinating your visit, keeping in mind that they are also managing a busy recruitment season. Postponing reimbursements until the conclusion of the job market might lead to cash flow challenges.
\item During a typical flyout, you will interact with the faculty from the Accounting department, their PhD students, and potentially the dean or deputy dean. Occasionally, you might also meet one or two faculty members from closely related departments, most commonly Finance. Expect the host department to provide you with a schedule for the flyout date. It is crucial to familiarize yourself with each individual you are scheduled to meet, including their research, personal interests, and other pertinent details. This preparation ensures you can engage in meaningful discussions lasting around 30 minutes with each person. I believe it is beneficial to acquaint yourself with all faculty members in the Accounting department, even if they are not listed on your itinerary. Unexpected changes can occur, or you might bump into them in passing and have an impromptu conversation.
\begin{itemize}
\item In your meeting with the department head, it is both appropriate and insightful to inquire about research budgets/resources and teaching loads.
\end{itemize}
\item - You can inquire about their decision-making timeline with the department head or with the recruitment committee chair.
\item Based on my personal experience, and I could be mistaken, if the person does not mention your job market paper during these office visits, do not try to bring it up. Remember, the workshop is designated for discussing your paper. Utilize your 30-minute interactions judiciously. The aim is for colleagues to perceive you as a pleasant future collaborator and someone they would enjoy spending time with.
\begin{itemize}
\item Be prepared for inquiries about your other papers and ensure you express genuine enthusiasm when discussing them. Conversations might revolve solely around these papers, and that is perfectly fine!
\end{itemize}
\item Your schedule will consist of consecutive meetings with faculty and deans, often without breaks in between. Typically, after your discussion with one person, they will escort you directly to the next individual's office.
\begin{itemize}
\item Monitor your liquid consumption! While it is essential to stay hydrated, avoid excessive drinking to minimize restroom visits.
\item Junior faculty, having recently gone through this process themselves, are often quite understanding. They might offer you a chance to take a restroom break. Do not hesitate if you feel the need. Even if you do not require a restroom visit, it might be a good opportunity to adjust your attire.
\end{itemize}
\item Naturally, you will be expected to present your job market paper, unless the host department requests a different paper. The allocated presentation time varies by department, typically ranging from 75 to 90 minutes. Hence, tailor your presentation to fit within this time frame. It might be beneficial to customize your presentation for each institution.
\begin{itemize}
\item Quite frequently, you might feel the pressure of time constraints, but this can be seen as a positive sign of an engaging presentation. While you should not stress excessively about variations in allotted time, remain vigilant. Even the longer durations can feel insufficient if not managed wisely.
\item It is crucial to be well-versed with the content of each slide and have smooth transitions planned between them.
\item Familiarize yourself with the number and order of your slides. Being able to quickly navigate to a specific slide and then revert to your original position can be both helpful and impressive to the audience.
\item Always ensure you conclude your presentation within the allotted time. Most likely that extending beyond the scheduled end time would count against you more than any additional information might benefit you. This is not only about conveying the content, but also about showcasing your ability to manage a presentation or even a class effectively. If faced with a question that requires a detailed response, it is acceptable to defer it momentarily for the sake of presentation flow. However, if multiple audience members pose similar questions, it indicates a potential gap in your presentation that needs addressing. Don't hesitate to skip slides to answer a question, but ensure you can seamlessly return to your original spot. Always maintain control of the presentation.
\end{itemize}
\item Although it might be challenging, when someone poses a question, prioritize understanding the query over formulating an immediate response. Ensure you have grasped the question accurately, and don not hesitate to repeat it or seek clarification. Misunderstanding a question can leave a negative impression.
\begin{itemize}
\item Begin your response concisely before delving into detailed explanations.
\item It is acceptable to admit when you are uncertain about an answer and express a willingness to consider it further. However, such admissions should be infrequent, ideally no more than twice during a presentation. Some questions are designed to assess the depth of your understanding and thoroughness.
\item Hostile questions are a rarity on the job market. Regardless of the nature of the inquiry, always maintain a courteous demeanor.
\end{itemize}
\item Again, people want to see you take their questions and concerns seriously. While you have the option to jot down notes regarding the questions posed, doing so might be distracting and time-consuming. Alternatively, if possible, consider requesting a PhD student to take notes on your behalf. Remember to express your gratitude to the student for their assistance.
\item Be prepared for potential technical glitches. Carry a USB drive with your presentation, and also save a copy in the cloud and your email for backup. However, unforeseen issues might prevent you from accessing your slides or displaying them. Ensure you are adequately prepared to deliver your presentation without visual aids. Trust me, if you have adequately prepared, you will be familiar with every slide. Being able to present under such circumstances could even earn you bonus points.
\end{itemize}

\section{Offers}
\label{sec:orgc64d659}
\begin{itemize}
\item Should you receive an offer, the call typically comes from either the department head or the recruitment committee chair. Congratulations! This is usually followed by an official offer letter.
\item In that initial call, they might not specify the exact salary figure. This detail might be reserved for the official letter. Some institutions might indicate that the formal letter will only be sent upon your acceptance. In such cases, it is entirely reasonable to inquire about the salary.
\item Remember, both your peers and the institutions are making critical decisions. If you receive a more appealing offer and intend to decline another, do so promptly. This allows institutions to extend offers to other candidates in a timely manner.
\begin{itemize}
\item If you have received a standout offer from School A and are content with it, I would suggest informing other institutions. Express gratitude for their consideration and clarify that you are stepping back from their recruitment process. Be cautious with your phrasing, especially if awaiting decisions from multiple institutions. The academic world is interconnected, and word can spread quickly. Avoid absolute statements like ``I will join School A'' if you are still considering other offers.
\end{itemize}
\item Some institutions may present ``exploding'' offers, which require swift decisions. Your reaction to such offers should be based on your unique circumstances. Consult with your advisors and close ones. If feasible, it is beneficial to wait for all decisions before committing.
\item It is crucial to be forthright during salary negotiations. While it is legitimate to negotiate if you are in a position to do so, always remain genuine. Only engage in talks with institutions you are genuinely considering. Having said that, you can absolutely negotiate with a school you are keen on joining, if approached tactfully.
\item Certain schools might organize a ``house hunting'' trip for you (and potentially your family) if you are seriously contemplating their offer. Only partake in these visits if you are genuinely inclined towards joining.
\end{itemize}

\newpage
\singlespace
\printbibliography
\end{document}